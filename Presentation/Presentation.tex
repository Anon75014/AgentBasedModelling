%-------------------------------------------------------
% HEADER
%-------------------------------------------------------

\documentclass[10pt, aspectratio=169]{beamer}
\usetheme[]{Feather}
 %Change the bar colors:
%\setbeamercolor{Feather}{fg=green!20,bg=green}
% Change the color of the structural elements:
%\setbeamercolor{structure}{fg=red}
% Change the frame title text color:
%\setbeamercolor{frametitle}{fg=blue}
% Change the normal text color background:
%\setbeamercolor{normal text}{fg=black,bg=gray!10}
\usepackage[utf8]{inputenc} %default packages
\usepackage[english]{babel}
\usepackage[T1]{fontenc}
\usepackage[scaled]{helvet}
\usefonttheme[onlymath]{serif} %For LaTeX-style looking equations
\usepackage{amsfonts} %default packages
\usepackage{amsmath}
\usepackage{amssymb}
\usepackage{amsthm}
\usepackage{amstext}
\usepackage{tikz} %making graphs within LaTeX
\usepackage{graphicx}
\usepackage{mathtools} %drawing picture with tikz
\usepackage{pgfplots} %making plots within LaTeX
\usepackage{pgfplotstable} %for importing data from csv files
\pgfplotsset{compat=1.17}
\usepackage{tabularx} %for tables centered
\usepackage{booktabs} %for better looking tables
\usepackage{longtable}
\usepackage{subcaption} % provides subfigures within figures 
\usepackage[absolute,overlay]{textpos} %for placing figures wherever
\usepackage{animate} %for making GIF animations
%small font size for the bibliography at the end
\newcommand\Fontvi{\fontsize{6}{7.2}\selectfont}
\usepackage{float} % Package to place figures where you want them.

\newcommand{\tagg}[1]{%
	\tikz[baseline]\node[anchor=base,
	draw=gray!30,
	rounded corners,
	inner xsep=1ex,
	inner ysep =0.75ex,
	text height=1.5ex,
	text depth=.25ex]{#1};
  }


%pgfplots style list (color and style of the plots for each farmer)
\definecolor{farmer1}{HTML}{f28d11} %orange
\definecolor{farmer2}{HTML}{0e5de8} %blue
\definecolor{farmer3}{HTML}{2bd941} %green
\definecolor{farmer4}{HTML}{d41717} %red
\definecolor{crop1}{HTML}{e41a1c} %
\definecolor{crop2}{HTML}{377eb8} %
\definecolor{crop3}{HTML}{4daf4a} %
\definecolor{crop4}{HTML}{984ea3} %
\pgfplotscreateplotcyclelist{farmerslist}{
%Farmer 1
farmer1,every mark/.append style={farmer1},mark=triangle*\\
farmer2,every mark/.append style={farmer2},mark=diamond*\\
farmer3,every mark/.append style={farmer3},mark=otimes\\
farmer4,every mark/.append style={farmer4},mark=star\\
%etc...
}
\pgfplotscreateplotcyclelist{cropslist}{
crop1, thick, every mark/.append style={crop1}, mark=triangle\\
crop2,every mark/.append style={crop2},mark=diamond\\
crop3,every mark/.append style={crop3},mark=otimes\\
crop4, densely dashed,every mark/.append style={crop4},mark=square*\\
%etc...
}


%-------------------------------------------------------
% INFORMATION IN THE TITLE PAGE
%-------------------------------------------------------

\title[] % [] is optional - is placed on the bottom of the sidebar on every slide
{ % is placed on the title page
      \textbf{851-0101-86 S Complex Social Systems: Modeling Agents, Learning, and Games}
}

\subtitle[]
{
      \textbf{CropWar: Agent-based simulation of agricultural interactions}
}

\author[]
{      C.Golling, G.Mourouga, A.Moser, O.Schmidt, A.H.Keshavarzzadeh
      {}
}

\institute[]
{
      %GDR redox-flow
  
  %there must be an empty line above this line - otherwise some unwanted space is added between the university and the country (I do not know why;( )
}

\date{}

%Programatically defining the sections to be called on the slides title
\def\z{Introduction and motivation}
\def\a{The CropWar model}
\def\aa{Deterministic versions}
\def\aaa{Basic version: selling and stocking}
\def\aab{The Map class: expansion}
\def\aac{The Market model: dynamic pricing}
\def\aad{Agent personalities}
\def\aae{Weather events}
\def\ab{Reinforcement Learning versions}
\def\aba{Introduction}
\def\abb{Proximal Policy Optimization}
\def\abc{ML agent trained against introverts}
\def\abd{ML agent trained against traders}
\def\b{Summary and Outlook}
\def\ba{Crop War in perspective}
\def\baa{FAO-based crop model}
\def\bab{Global warming potential}

%-------------------------------------------------------
% START OF DOCUMENT
%-------------------------------------------------------

\begin{document}

%-------------------------------------------------------
% THE TITLEPAGE
%-------------------------------------------------------

{\1% % this is the name of the PDF file for the background
\begin{frame}[plain,noframenumbering] % the plain option removes the header from the title page, noframenumbering removes the numbering of this frame only
  \titlepage % call the title page information from above
\end{frame}}


\begin{frame}{Content}{}
    \tableofcontents
\end{frame}

%-------------------------------------------------------
\section{\z}
%-------------------------------------------------------

\begin{frame}{\z}
  \onslide<1->{\texttt{CropWar} aims at modelling \textbf{interactions between farmers} growing crops in a \textbf{geographic area} with a finite amount of \textbf{water ressources} and selling them on a \textbf{market} to feed a population.}\\
 \onslide<2->{The model is organised as follows:}
 \begin{itemize}
  \item<3-> \tagg{v1.1} basic version: growing, selling, stocking
  \item<4-> \tagg{v1.2} implementation of the \texttt{Map} class: spatial expansion
  \item<5-> \tagg{v1.3} implementation of the \texttt{Market} model: dynamic pricing
  \item<6-> \tagg{v1.4} implementation of agent personalities: game theory
  \item<7-> \tagg{v2.1} reinforcement learning: identifying the optimal strategy
\end{itemize}
\end{frame}

%-------------------------------------------------------
\section{\z}
%-------------------------------------------------------

\subsection{\aa}
%-------------------------------------------------------

\subsubsection{\aaa}
\begin{frame}{\aa}{\aaa}
  \centering
  \textbf{v1.1: Basic version}
\end{frame}


\begin{frame}{\aa}{\aaa}
    \begin{columns}
      \column{0.48\linewidth}
         \centering
         \begin{figure}
          \includegraphics[width=\textwidth]{Figures/v11_Map.png}
         \end{figure}
       \column{0.52\linewidth}
       In \tagg{v1.1} farmers only have access to one cell on which they can choose to grow one of two crops (A or B)
     \end{columns} 
\end{frame}

\pgfplotstableread[col sep=comma]{Data/v11_exported_budget.csv}{\budgeta}
\pgfplotstableread[col sep=comma, skip first n=3]{Data/v11_exported_stock.csv}{\stocka}

\begin{frame}{\aa}{\aaa}
  \begin{columns}
    \column{0.48\linewidth}
       \centering
       \begin{figure}
       \resizebox{\linewidth}{!}{\begin{tikzpicture}
  \pgfplotsset{
    scale only axis,
  }
  %axis options:
  \begin{axis}[
    xlabel=Time step (years),
    ylabel=Farmer stock,
    xmin=0,
    %ymin=350,
    legend pos=north west,
    cycle list name=farmerslist
  ]
    %plot 1: farmer ID 1
    \addplot+[] table[y index =1]{\stocka};
    \addlegendentry{$1$}
    
    %plot 2: farmer ID 2
    \addplot+[]  table[y index =2]{\stocka};
    \addlegendentry{$2$}

    %plot 3: farmer ID 3
    \addplot+[dashed] table[y index =7]{\stocka};
    \addlegendentry{$3$}

    %plot 4: farmer ID 4
    \addplot+[] table[y index =4]{\stocka};
    \addlegendentry{$4$}

  \end{axis}
  \end{tikzpicture}
  }
       \end{figure}
     \column{0.52\linewidth}
     This plot shows the evolution of stock as a function of time:\\
     Farmers 1,2 and 4 grow crop A\\
     Farmer 3 grows crop B\\
     At each time step, they chose to stock or sell their yield
   \end{columns} 
\end{frame}

\begin{frame}{\aa}{\aaa}
    \begin{columns}
      \column{0.48\linewidth}
         \centering
         \begin{figure}
         \resizebox{\linewidth}{!}{\begin{tikzpicture}
		\pgfplotsset{
			scale only axis,
		}
		%axis options:
		\begin{axis}[
		  xlabel=Time step (years),
		  ylabel=Farmer wealth (\$),
		  xmin=0,
		  %ymin=350,
		  legend pos=north west,
		  cycle list name=farmerslist
		]
		  %plot 1: farmer ID 1
		  \addplot+[] table[y index =1]{\budgeta};
		  \addlegendentry{$1$}
		  
		  %plot 2: farmer ID 2
		  \addplot+[]  table[y index =2]{\budgeta};
		  \addlegendentry{$2$}
  
		  %plot 3: farmer ID 3
		  \addplot+[] table[y index =3]{\budgeta};
		  \addlegendentry{$3$}
  
		  %plot 4: farmer ID 4
		  \addplot+[] table[y index =4]{\budgeta};
		  \addlegendentry{$4$}
  
		\end{axis}
	  \end{tikzpicture}
	  }
         \end{figure}
       \column{0.52\linewidth}
       This plot shows the corresponding evolution of the farmer's wealth as a function of time.\\
       Selling corresponds to an increase in wealth, stocking to a constant value.
     \end{columns} 
\end{frame}

\subsubsection{\aab}

\begin{frame}{\aa}{\aab}
  \centering
  \textbf{v1.2: Spatial expansion}
\end{frame}

\begin{frame}{\aa}{\aab}
  \begin{columns}
    \column{0.48\linewidth}
    \centering
    \begin{figure}
      \includegraphics[width=.8\textwidth]{Figures/v12_Map_start.png}
     \end{figure}
    \column{0.52\linewidth}
     The \texttt{buy\_threshold} (in brackets) indicates the tendancy of farmers to expand their farms by buying neighbouring land in order to grow more crops
  \end{columns}
\end{frame}

\begin{frame}{\aa}{\aab}
  \begin{columns}
    \column{0.48\linewidth}
    \centering
    \begin{figure}
      \includegraphics[width=.8\textwidth]{Figures/v12_Map_start.png}
     \end{figure}
    \column{0.52\linewidth}
    \centering
    \animategraphics[loop,width=5.8cm]{1}{Figures/V12mapGIF/frame_}{00}{10}
  \end{columns}
\end{frame}

\pgfplotstableread[col sep=comma]{Data/v12_exported_budget.csv}{\budgetb}
\pgfplotstableread[col sep=comma, skip first n=3]{Data/v12_exported_stock.csv}{\stockb}
\pgfplotstableread[col sep=comma, skip first n=1]{Data/v12_exported_cellcount.csv}{\cellcountb}

\begin{frame}{\aa}{\aab}
  \begin{columns}
    \column{0.48\linewidth}
    \resizebox{\linewidth}{!}{\begin{tikzpicture}
      \pgfplotsset{
        scale only axis,
      }
      %axis options:
      \begin{axis}[
        xlabel=Time step (years),
        ylabel=Farmer land count (cells),
        xmin=0,
        legend pos=north west,
        cycle list name=farmerslist
      ]
        %plot 1: farmer ID 1
        \addplot+[] 
          table[y index =1]{\cellcountb};
        \addlegendentry{$1$}
        
        %plot 2: farmer ID 2
        \addplot+[]  table[y index =2]{\cellcountb};
        \addlegendentry{$2$}
    
        %plot 3: farmer ID 3
        \addplot+[] table[y index =3]{\cellcountb};
        \addlegendentry{$3$}
    
        %plot 4: farmer ID 4
        \addplot+[] table[y index =4]{\cellcountb};
        \addlegendentry{$4$}
    
      \end{axis}
      \end{tikzpicture}
    }
    \column{0.52\linewidth}
    \resizebox{\linewidth}{!}{\begin{tikzpicture}
      \pgfplotsset{
        scale only axis,
      }
      
      \begin{axis}[
        xlabel=Time step (years),
        ylabel=Farmer stock,
        xmin=0,
        ymin=0,
        legend pos=north west,
        cycle list name=farmerslist
      ]
        %plot 1: farmer ID 1
      \addplot+[dashed] table[y index =5]{\stockb};
      \addlegendentry{$1$}
      
      %plot 2: farmer ID 2
      \addplot+[]  table[y index =2]{\stockb};
      \addlegendentry{$2$}
  
      %plot 3: farmer ID 3
      \addplot+[] table[y index =3]{\stockb};
      \addlegendentry{$3$}
  
      %plot 4: farmer ID 4
      \addplot+[] table[y index =4]{\stockb};
      \addlegendentry{$4$}
  
      \end{axis}
      \end{tikzpicture}
    }
  \end{columns}
\end{frame}

\begin{frame}{\aa}{\aab}
  \begin{columns}
    \column{0.48\linewidth}
    \centering
    \begin{figure}
      \resizebox{\linewidth}{!}{\begin{tikzpicture}
        \pgfplotsset{
          scale only axis,
        }
        %axis options:
        \begin{axis}[
          xlabel=Time step (years),
          ylabel=Farmer wealth (\$),
          xmin=0,
          ymax = 600,
          legend pos=north west,
          cycle list name=farmerslist
        ]
          %plot 1: farmer ID 1
          \addplot+[] 
            table[y index =1]{\budgetb};
          \addlegendentry{$1$}
          
          %plot 2: farmer ID 2
          \addplot+[]  table[y index =2]{\budgetb};
          \addlegendentry{$2$}
      
          %plot 3: farmer ID 3
          \addplot+[] table[y index =3]{\budgetb};
          \addlegendentry{$3$}
      
          %plot 4: farmer ID 4
          \addplot+[] table[y index =4]{\budgetb};
          \addlegendentry{$4$}
      
        \end{axis}
        \end{tikzpicture}
        }
     \end{figure}
    \column{0.52\linewidth}
    Farmer 2 seems to remain wealthier than other farmers over 10 time steps, due to not spending money to expand
  \end{columns}
\end{frame}

\pgfplotstableread[col sep=comma]{Data/v12_exported_budget_50t.csv}{\budgetc}
\pgfplotstableread[col sep=comma, skip first n=3]{Data/v12_exported_stock_50t.csv}{\stockc}
\pgfplotstableread[col sep=comma, skip first n=1]{Data/v12_exported_cellcount_50t.csv}{\cellcountc}

\begin{frame}{\aa}{\aab}
  \begin{columns}
    \column{0.48\linewidth}
    \centering
    \begin{figure}
      \includegraphics[width=.8\textwidth]{Figures/v12_Map_50t.png}
     \end{figure}
    \column{0.52\linewidth}
    If we extend the simulation to 50 time steps, all available land is bought by farmers
  \end{columns}
\end{frame}

\begin{frame}{\aa}{\aab}
  \begin{columns}
    \column{0.48\linewidth}
    \resizebox{\linewidth}{!}{\begin{tikzpicture}
      \pgfplotsset{
        scale only axis,
      }
      %axis options:
      \begin{axis}[
        xlabel=Time step (years),
        ylabel=Farmer land count (cells),
        xmin=0,
        legend pos=north west,
        cycle list name=farmerslist
      ]
        %plot 1: farmer ID 1
        \addplot+[] 
          table[y index =1]{\cellcountc};
        \addlegendentry{$1$}
        
        %plot 2: farmer ID 2
        \addplot+[]  table[y index =2]{\cellcountc};
        \addlegendentry{$2$}
    
        %plot 3: farmer ID 3
        \addplot+[] table[y index =3]{\cellcountc};
        \addlegendentry{$3$}
    
        %plot 4: farmer ID 4
        \addplot+[] table[y index =4]{\cellcountc};
        \addlegendentry{$4$}
    
      \end{axis}
      \end{tikzpicture}
    }
    \column{0.52\linewidth}
    \resizebox{\linewidth}{!}{\begin{tikzpicture}
      \pgfplotsset{
        scale only axis,
      }
      
      \begin{axis}[
        xlabel=Time step (years),
        ylabel=Farmer stock,
        xmin=0,
        ymin=0,
        legend pos=north west,
        cycle list name=farmerslist
      ]
        %plot 1: farmer ID 1
      \addplot+[dashed] table[y index =5]{\stockc};
      \addlegendentry{$1$}
      
      %plot 2: farmer ID 2
      \addplot+[]  table[y index =2]{\stockc};
      \addlegendentry{$2$}
  
      %plot 3: farmer ID 3
      \addplot+[] table[y index =3]{\stockc};
      \addlegendentry{$3$}
  
      %plot 4: farmer ID 4
      \addplot+[] table[y index =4]{\stockc};
      \addlegendentry{$4$}
  
      \end{axis}
      \end{tikzpicture}
    }
  \end{columns}
\end{frame}

\begin{frame}{\aa}{\aab}
  \begin{columns}
    \column{0.48\linewidth}
    \centering
    \begin{figure}
      \resizebox{\linewidth}{!}{\begin{tikzpicture}
        \pgfplotsset{
          scale only axis,
        }
        %axis options:
        \begin{axis}[
          xlabel=Time step (years),
          ylabel=Farmer wealth (\$),
          xmin=0,
          legend pos=north west,
          cycle list name=farmerslist
        ]
          %plot 1: farmer ID 1
          \addplot+[] 
            table[y index =1]{\budgetc};
          \addlegendentry{$1$}
          
          %plot 2: farmer ID 2
          \addplot+[]  table[y index =2]{\budgetc};
          \addlegendentry{$2$}
      
          %plot 3: farmer ID 3
          \addplot+[] table[y index =3]{\budgetc};
          \addlegendentry{$3$}
      
          %plot 4: farmer ID 4
          \addplot+[] table[y index =4]{\budgetc};
          \addlegendentry{$4$}
      
        \end{axis}
        \end{tikzpicture}
        }
     \end{figure}
    \column{0.52\linewidth}
    On the long run, farmers with more land become wealthier, although the type of crop also seems to have an importance
  \end{columns}
\end{frame}

\begin{frame}{\aa}{\aac}
  \centering
  \textbf{v1.3: Dynamic pricing of crops}
\end{frame}

\begin{frame}{\aa}{\aac}
 \onslide<1->{In \tagg{v1.3} we implement a \texttt{Market} class, which updates the price of crops based on supply and demand.}\\
 \begin{enumerate}
    \item<2-> Harvesting period: agents harvest according to their crop choice and add the harvest yield to the stock. 
    \item<3->Interaction period: market interaction takes place.
    \item<4-> Strategy period: the trader type of the farmers will react on market prices and eventually change the crop.
\end{enumerate}
\end{frame}

\begin{frame}{\aa}{\aac}
    Supply of crop $j$ for a price $p_j$ grown by farmer $i$ is considered to depend on the willingness of the agent to supply crops at a given price:
    \begin{equation}
      S_i(p_j) = A + c_i \cdot p_j
    \end{equation}
  Here $c_i$ is the willingness to supply crop $j$  at a given price $p_j$ by farmer $i$ and $A$ is the surplus or his willingness to supply crops for free.
\end{frame}

\begin{frame}{\aa}{\aac}
Demand is considered to be a function of the population growth and the price:
  \begin{equation}
    D_j(p_j) = \left( D_0 + at^2 \right) e^{-\alpha (p_j-p_{j0})}
  \end{equation}
Where $D_0$ is a base demand, $at^2$ is a term representing population (and demand) growth as a function of time $t$, $p_j$ is the price of crop $j$ at time $t$, $p_{j,0}$ is the base price of crop $j$, and $\alpha$ is a demand slope, representing the fact that less and less people will be able to afford expensive crops.
\end{frame}

\pgfplotstableread[col sep=comma]{Data/v13_exported_budget.csv}{\budgetc}
\pgfplotstableread[col sep=comma, skip first n=3]{Data/v13_exported_stock.csv}{\stockc}
\pgfplotstableread[col sep=comma, skip first n=1]{Data/v13_exported_cellcount.csv}{\cellcountc}
\pgfplotstableread[col sep=comma]{Data/v13_exported_demand.csv}{\demandc}
\pgfplotstableread[col sep=comma]{Data/v13_exported_prices.csv}{\pricesc}
\pgfplotstableread[col sep=comma]{Data/v13_exported_supply.csv}{\supplyc}
\pgfplotstableread[col sep=comma]{Data/v13_exported_global_stock.csv}{\globalstockc}


\begin{frame}{\aa}{\aac}
  \begin{columns}
    \column{0.5\linewidth}
    \resizebox{\linewidth}{!}{\begin{tikzpicture}
		\pgfplotsset{
			scale only axis,
		}
		%axis options:
		\begin{axis}[
		  xlabel=Time step (years),
		  ylabel=Demand for different crops,
		  xmin=0,
		  xmax=30,
		  legend pos=south east,
		  cycle list name=cropslist
		]
		  %plot 1: farmer ID 1
		  \addplot+[] 
			  table[y index =1]{\demandc};
		  \addlegendentry{$1$}
		  
		  %plot 2: farmer ID 2
		  \addplot+[]  table[y index =2]{\demandc};
		  \addlegendentry{$2$}
  
		  %plot 3: farmer ID 3
		  \addplot+[] table[y index =3]{\demandc};
		  \addlegendentry{$3$}
  
		  %plot 4: farmer ID 4
		  \addplot+[] table[y index =4]{\demandc};
		  \addlegendentry{$4$}
  
		\end{axis}
	  \end{tikzpicture}
	}
    \column{0.5\linewidth}
    \resizebox{\linewidth}{!}{\begin{tikzpicture}
		\pgfplotsset{
			scale only axis,
		}
		%axis options:
		\begin{axis}[
		  xlabel=Time step (years),
		  ylabel=Supply for different crops,
		  xmin=0,
		  xmax=30,
		  legend pos=south east,
		  cycle list name=cropslist
		]
		  %plot 1: farmer ID 1
		  \addplot+[] 
			  table[y index =1]{\supplyc};
		  \addlegendentry{$1$}
		  
		  %plot 2: farmer ID 2
		  \addplot+[]  table[y index =2]{\supplyc};
		  \addlegendentry{$2$}
  
		  %plot 3: farmer ID 3
		  \addplot+[] table[y index =3]{\supplyc};
		  \addlegendentry{$3$}
  
		  %plot 4: farmer ID 4
		  \addplot+[] table[y index =4]{\supplyc};
		  \addlegendentry{$4$}
  
		\end{axis}
	  \end{tikzpicture}
	  }
  \end{columns}
\end{frame}

\begin{frame}{\aa}{\aac}
\begin{columns}
    \column{0.5\linewidth}
    \resizebox{\linewidth}{!}{\begin{tikzpicture}
		\pgfplotsset{
			scale only axis,
		}
	  
		\begin{axis}[
		  xlabel=Time step (years),
		  ylabel=Prices of different crops,
		  xmin=0,
		  ymin=0,
		  legend pos=north west,
		  cycle list name=cropslist
		]
		  %plot 1: farmer ID 1
	  \addplot+[] table[y index =1]{\pricesc};
	  \addlegendentry{$1$}
	  
	  %plot 2: farmer ID 2
	  \addplot+[]  table[y index =2]{\pricesc};
	  \addlegendentry{$2$}

	  %plot 3: farmer ID 3
	  \addplot+[] table[y index =3]{\pricesc};
	  \addlegendentry{$3$}

	  %plot 4: farmer ID 4
	  \addplot+[] table[y index =4]{\pricesc};
	  \addlegendentry{$4$}

	  \end{axis}
	  \end{tikzpicture}
  }
    \column{0.5\linewidth}
    \resizebox{\linewidth}{!}{\begin{tikzpicture}
		\pgfplotsset{
			scale only axis,
		}
	  
		\begin{axis}[
		  xlabel=Time step (years),
		  ylabel=Total stock of different crops,
		  xmin=0,
		  ymin=0,
		  legend pos=north west,
		  cycle list name=cropslist
		]

		  \addplot+[] table[y index =1]{\globalstockc};
		  \addlegendentry{$1$}
		  \addplot+[] table[y index =2]{\globalstockc};
		  \addlegendentry{$2$}
		  \addplot+[] table[y index =3]{\globalstockc};
		  \addlegendentry{$3$}
		  \addplot+[] table[y index =4]{\globalstockc};
		  \addlegendentry{$4$}
		\end{axis}
	  \end{tikzpicture}
  }
  \end{columns}
\end{frame}

\begin{frame}{\aa}{\aac}
\begin{columns}
    \column{0.45\linewidth}
    \resizebox{\linewidth}{!}{
		\includegraphics[width=\textwidth]{Figures/v13_Map.png}
	}
    \column{0.5\linewidth}
    \resizebox{\linewidth}{!}{\begin{tikzpicture}
		\pgfplotsset{
			scale only axis,
		}
		%axis options:
		\begin{axis}[
		  xlabel=Time step (years),
		  ylabel=Farmer wealth (\$),
		  xmin=0,
		  %ymin=350,
		  legend pos=north west,
		  cycle list name=farmerslist
		]
		  %plot 1: farmer ID 1
		  \addplot+[] table[y index =1]{\budgetc};
		  \addlegendentry{$1$}
		  
		  %plot 2: farmer ID 2
		  \addplot+[]  table[y index =2]{\budgetc};
		  \addlegendentry{$2$}
  
		  %plot 3: farmer ID 3
		  \addplot+[] table[y index =3]{\budgetc};
		  \addlegendentry{$3$}
  
		  %plot 4: farmer ID 4
		  \addplot+[] table[y index =4]{\budgetc};
		  \addlegendentry{$4$}
  
		\end{axis}
	  \end{tikzpicture}
	  }
  \end{columns}
\end{frame}

\subsection{\ab}
\subsubsection{\aba}

\begin{frame}{\ab}{\aba}
  \centering
  \textbf{Reinforcement Learning}
\end{frame}

\begin{frame}{\ab}{\aba}
  \centering
The preceding versions \tagg{v1.1 - v1.3} defined the deterministic agents and the model of the environment in which actions take place.

In order to see unpredictable, emerging behaviour, the agents would need to adjust to evolutions of the market and learn.
\end{frame}


\begin{frame}{\ab}{\aba}
  \centering
    \begin{figure}
      \includegraphics[width = .8\textwidth]{Figures/RL schematic.PNG}
      \caption*{\tiny Feedback loop in RL. An agent decides to do a certain action $a_t$ based on his observation of the environment. This affects the environment, which yields a new state. The effect of the agents' action is then interpreted to update the strategy of the agent.}
    \end{figure}
\end{frame}

\subsubsection{\abb}


\begin{frame}{\ab}{\abb}
\onslide<1->{\begin{equation}
  \underbrace{L(\theta)}_{\substack{\text{Policy}\\ \text{loss}\\ \text{function}}} = \underbrace{\hat{\mathbf{E}}_t}_{\substack{\text{expectation}\\ \text{at iteration}\\ t}} \left[ \log ( \underbrace{\pi_{\theta}(a_t|s_t))}_{\substack{\text{stochastic}\\ \text{policy}}} \underbrace{\hat{A_t}}_{\substack{\text{estimator of}\\ \text{advantage of} \\ \text{current action} }} \right]
\end{equation}}
\onslide<1->{where:}
\begin{itemize}
  \item<3-> $\pi_{\theta}(a_t|s_t)$ is given as transition probabilities in the MC of taking action $a_t$ in state $s_t$. It is a neural network, that suggests an action for a given state based on previous training experience.
  \item<4-> $\hat{A_t}$ is computed as a discounted reward. This neural net is updated with the experience (i.e. reward) that the agent collects in an environment.
\end{itemize}
\end{frame}

\begin{frame}{\ab}{\abb}
\begin{equation}
  \hat{g} = \hat{\mathbf{E}}_t \left[ \nabla_{\theta} \log(\pi_{\theta}(a_t|s_t)) \hat{A_t} \right]
\end{equation}
This would increase the probability of taking the same action decision in the same or similar state. Vice versa, the gradient is negative if the advantage is negative, which reduces the probability of taking the corresponding action.
\end{frame}


\subsubsection{\abc}

\begin{frame}{\ab}{\abc}
  \centering
  \textbf{v2.1: RL agent trained againt Introvert farmers}
\end{frame}

\begin{frame}{\ab}{\abc}
  The reward function that was used for training is given by
\begin{equation}
    \frac{1}{3}(b^2 + r^2 + s^2)
\end{equation}
where $b$ is the fraction of the global budget that the agent possesses, $r$ is the current ranking (1 for first, $0.66$ for second, $0.33$ for third, and
$0.0$ for last) and $s$ is the fraction of the global supply that the agent
supplies. It was chosen so to maximize profit of each agent.
\end{frame}

\pgfplotstableread[col sep=comma]{Data/MLvsIntroBudget.csv}{\budgetd}

\begin{frame}{\ab}{\abc}
\begin{figure}[H]
    \centering
    \resizebox{.55\linewidth}{!}{\begin{tikzpicture}
		\pgfplotsset{
			scale only axis,
		}
		%axis options:
		\begin{axis}[
		  xlabel=Time step (years),
		  ylabel=Farmer wealth (\$),
		  xmin=0,
		  %ymin=350,
		  legend pos=north west,
		  cycle list name=farmerslist
		]
		  %plot 1: farmer ID 1
		  \addplot+[thick, no marks] table[y index =1]{\budgetd};
		  \addlegendentry{Introvert}
		  
		  %plot 2: farmer ID 2
		  \addplot+[thick,no marks]  table[y index =2]{\budgetd};
		  \addlegendentry{Introvert}
  
		  %plot 3: farmer ID 3
		  \addplot+[thick,no marks] table[y index =3]{\budgetd};
		  \addlegendentry{RL agent}
  
		  %plot 4: farmer ID 4
		  \addplot+[thick,no marks] table[y index =4]{\budgetd};
		  \addlegendentry{Introvert}
  
		\end{axis}
	  \end{tikzpicture}
	  }
    \caption{Budget evolution over 300 iterations}
    \label{fig:budget MLvsIntros}
\end{figure}
\end{frame}

\subsubsection{\abd}

\begin{frame}{\ab}{\abd}
  \centering
  \textbf{v2.1: RL agent trained againt Trader farmers}
\end{frame}

\pgfplotstableread[col sep=comma]{Data/MLvsTrader2Budget.csv}{\budgete}
\pgfplotstableread[col sep=comma]{Data/MLvsTrader2Cellcount.csv}{\cellcounte}

\begin{frame}{\ab}{\abd}
  \begin{columns}
      \column{0.5\linewidth}
      \resizebox{\linewidth}{!}{\begin{tikzpicture}
        \pgfplotsset{
          scale only axis,
        }
        %axis options:
        \begin{axis}[
          xlabel=Time step (years),
          ylabel=Farmer wealth (\$),
          xmin=0,
          %ymin=350,
          legend pos=south west,
          cycle list name=farmerslist
        ]
          %plot 1: farmer ID 1
          \addplot+[] table[y index =1]{\budgete};
          \addlegendentry{Trader}
          
          %plot 2: farmer ID 2
          \addplot+[]  table[y index =2]{\budgete};
          \addlegendentry{RL agent}
      
          %plot 3: farmer ID 3
          \addplot+[] table[y index =3]{\budgete};
          \addlegendentry{Trader}
      
          %plot 4: farmer ID 4
          \addplot+[] table[y index =4]{\budgete};
          \addlegendentry{Trader}
      
        \end{axis}
        \end{tikzpicture}
        }

    \column{0.5\linewidth}
      \resizebox{\linewidth}{!}{\begin{tikzpicture}
        \pgfplotsset{
          scale only axis,
        }
        %axis options:
        \begin{axis}[
          xlabel=Time step (years),
          ylabel=Farmer land count (cells),
          xmin=0,
          legend pos=south east,
          cycle list name=farmerslist
        ]
          %plot 1: farmer ID 1
          \addplot+[] 
            table[y index =1]{\cellcounte};
          \addlegendentry{Trader}
          
          %plot 2: farmer ID 2
          \addplot+[]  table[y index =2]{\cellcounte};
          \addlegendentry{RL agent}
      
          %plot 3: farmer ID 3
          \addplot+[] table[y index =3]{\cellcounte};
          \addlegendentry{Trader}
      
          %plot 4: farmer ID 4
          \addplot+[] table[y index =4]{\cellcounte};
          \addlegendentry{Trader}
      
        \end{axis}
        \end{tikzpicture}
      }
    \end{columns}
  \end{frame}

  
  \begin{frame}{\ab}{\abd}
    \begin{figure}[H]
      \centering
      \includegraphics[width = 1\textwidth]{Figures/stock evolution.PNG}
      \caption{Stock evolution for the three \texttt{Trader} agents and the RL agent.}
      \label{fig:stock evolve}
  \end{figure}
  \end{frame}



%-------------------------------------------------------
\section{\b}
%-------------------------------------------------------

%%%%%%%%%%%%%%%%%%%%%%%%%%%%%%%%


\subsection{\ba}
\subsubsection{\baa}

\begin{frame}{\ba}
  \centering
  \textbf{Outlook and Perspective}
\end{frame}

\begin{frame}{\ba}{\baa}
    \begin{columns}
      \column{0.65\linewidth}
      \centering
        \begin{figure}
         \includegraphics[width=\textwidth]{Figures/Crop-coefficients-Kc-and-growing-period-of-tomato-source-Allen-et-al-1998.png}
         \label{fig:y equals x}
        \end{figure}
        \column{0.4\linewidth}
        comment
    \end{columns}
\end{frame}

\begin{frame}{\ba}{\baa}
  \begin{columns}
    \column{0.6\linewidth}
  \begin{figure}
    \centering
    \includegraphics[width=\textwidth]{Figures/Y3655E01.jpeg}
  \end{figure}
  \column{0.4\linewidth}
  \begin{equation}
    1-\frac{Y_t^c}{Y^{m.c}}=\sum_{i=1}^{n} k_{y.i} \cdot(1-\frac{ET_{a.t}^c}{ET_{m.i}}) 
  \end{equation}
\end{columns} 
\end{frame}

\subsubsection{\bab}

\begin{frame}{\ba}{\bab}
  \begin{figure}
    \centering
    \includegraphics[width=.5\textwidth]{Figures/Emissions-by-sector-–-pie-charts.png}
  \end{figure} 
\end{frame}


\begin{frame}{\ba}{\bab}
  \begin{figure}
    \centering
    \includegraphics[width=.6\textwidth]{Figures/GHG-emissions-from-agricultural-sector-by-practices-in-Mt-CO2-e}
  \end{figure}
  \begin{equation}
    GWP_{t}=GWP_{fertilizer}+GWP_{biocide}+GWP_{machinery}+GWP_{electricity}  
    \end{equation}
\end{frame}
 

\subsubsection{\aac}
\subsubsection{\aad}
\subsubsection{\aae}

\subsection{\ab}


\subsubsection{\abb}
\end{document}

